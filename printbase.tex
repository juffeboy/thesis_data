\documentclass[a4paper]{article}

%% Language and font encodings
\usepackage[english]{babel}
\usepackage[utf8x]{inputenc}
\usepackage[T1]{fontenc}

%% Sets page size and margins
\usepackage[a4paper,top=4cm,bottom=4cm,left=4cm,right=4 cm,marginparwidth=1.75cm]{geometry}

%% Useful packages
\usepackage{amsmath}
\usepackage{graphicx}
\usepackage{tikz}
\usepackage{pgfplots}
\usepackage{pgfplotstable}
\usepackage{multirow}
\pgfdeclarelayer{f}
\pgfsetlayers{main,f}
\usepackage[colorinlistoftodos]{todonotes}
\usepackage[colorlinks=true, allcolors=blue]{hyperref}
\usepackage[font={small,it}]{caption}

\title{Ping/Peek test}
\author{l}

\begin{document}
\maketitle

\begin{abstract}
%Tested to ping and peek directly to the sensor node with IPV6. 
%Both test were based on 100 sample tests.
%Every lefthand value in the figures start on 17 bytes(data, 65 byte with the udp header included), and the the resulting frame on the radio link is 93 bytes. the incoming frame is 71 for such a message according to the table.
%
%PING and CCN is there measured in bytes.
%The 17 byte CCN packet is containing a search string. ``s + mote1*X'' where the X is incremented to get a bigger packet size.
%
\end{abstract}


\newpage



%%%%% SHORT X - AXIS











%%%%%%%%%%%%

\end{document}